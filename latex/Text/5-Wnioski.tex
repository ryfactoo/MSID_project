\section{Wnioski}
% Delete the text and write your Discussion here:
%------------------------------------

Przedstawione eksperymenty wskazały, że nasze modele bardzo dobrze sobie radzą z klasyfikacją kategorii od parametrów filmów. Raport wskazuje również na zależność między wszystkimi atrybutami a kategorią. 

Warto również zauważyć, że nasze badania uwzględniały różne metody klasyfikacji, takie jak Random Forest Classifier \ref{RFC} oraz Decision Tree Classifier \ref{DTC}. Porównując wyniki obu modeli, stwierdziliśmy, że model RFC radzi sobie znacznie lepiej niż model DTC. Różnica w wynikach między nimi wynosi aż 9 punktów procentowych, co sugeruje, że model RFC jest bardziej efektywny w klasyfikacji kategorii filmów na podstawie parametrów, bo otrzymujemy wynik klasyfikacji na poziomie 82\%.

Na podstawie tych obserwacji, możemy wnioskować, że nasz model ma potencjał do dalszego rozwoju i zastosowania w przyszłych badaniach. Może być również używany do klasyfikacji innych zbiorów danych związanych z YouTube.